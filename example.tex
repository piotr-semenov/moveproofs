\documentclass{article}

\usepackage[location=appendix]{moveproofs}

% This is necessary: moveproofs will define a proof environment if you don't have one, 
% but you have to define your own theorem, corollary, etc. environments.
\newtheorem{theorem}{Theorem}
\newtheorem{corollary}[theorem]{Corollary}

\title{Example document with proofs moved around.}
\author{Daniel Haas}

\begin{document}

\maketitle

\section{Paper Body}
Let's prove some theorems! Because ``location" was set to \ifmoveproofstoappendix ``appendix'', \else ``inline'', \fi the proofs will appear \ifmoveproofstoappendix in the appendix. \else inline, and no appendix will be displayed. \fi


% Theorem one.
\begin{theorem}\label{theorem_one}
Given $\delta \in (0, 1)$, with probability $1 - \delta$, proofs are useful.
\end{theorem}

\makeproof{theorem_one}{
Proofs are great! Therefore, we have our result.
}

% Corollary two.
\vspace{0.4cm}
\noindent We can also use the moveproofs package to prove statements that aren't theorems.
Note the reference to the corollary when this proof is moved to the appendix.

\begin{corollary}\label{corollary_two}
Given $\delta \in (0, 1)$, with probability $1 - \delta$, corollaries are useful.
\end{corollary}

\makeproof{corollary_two}{
Corollaries support proofs. Therefore, we have our result.
}

% Appendix
\appendix
\appendixproofsection{Our Proofs}
\appendixproof{theorem_one}
\appendixproof{corollary_two}

\end{document}